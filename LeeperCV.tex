% LaTeX resume using res.cls
\documentclass[line,margin]{res}
\usepackage{ifthen}
\usepackage{tabularx,booktabs}
%\usepackage{helvetica} % uses helvetica postscript font (download helvetica.sty)
%\usepackage{newcent}   % uses new century schoolbook postscript font
\setlength{\textheight}{10in}
%\setlength{\textwidth}{7in}
\setlength{\voffset}{-0.5in}
%\setlength{\hoffset}{-0.5in}
\oddsidemargin=-0.5in
\resumewidth=7.5in

\newsectionwidth{20pt}

\newcommand{\ifthen}[2]{\ifthenelse{#1}{#2}{}}

%\newcommand{\compactlist}{
%  \itemize
%  \setlength{\itemsep}{0pt}%
%  \setlength{\parskip}{0pt}%
%%  \setlength{\parsep}{0pt}
%}

\newenvironment{compactlist}{
	\begin{itemize}\itemsep=0pt
}{
	\end{itemize}
}

%%% We can put stuff inside a \CVOnly{} tag to make it only show up for a CV :)
\providecommand{\ResumeCV}{0}% fallback definition % 0=Resume, 1=CV
%\def\ResumeCV{0}  % 0=Resume, 1=CV
\ifthen { \ResumeCV = 0 }
        { \newcommand{\ResumeOnly}[1]{#1}
          \newcommand{\CVOnly}[1]{}
          \newcommand{\CVOnlySmall}[1]{{\small \CVOnly{#1}}}
          \newcommand{\ResumeOnlySmall}[1]{{\small \CVOnly{#1}}}
          \newcommand{\BothSmall}[1]{{\small #1}}   }
\ifthen { \ResumeCV = 1 }
        { \newcommand{\ResumeOnly}[1]{}
          \newcommand{\CVOnly}[1]{#1}
          \newcommand{\CVOnlySmall}[1]{{\CVOnly{#1}}}
          \newcommand{\ResumeOnlySmall}[1]{{\CVOnly{#1}}}
          \newcommand{\BothSmall}[1]{{#1}}    }

\newcommand{\hide}[1]{}

\newcommand{\boldName}[1]{\textbf{#1}}

\begin{document}
%\resumewidth{100pt}

\name{Adam Eric Leeper}
% \address used twice to have two lines of address
\address{469 Homer Ave, Palo Alto, CA 94301}
\address{650.762.6844}
\address{adamleeper@gmail.com}
\address{adamleeper.com $\mid$ github.com/aleeper}

%\setpapersize{USletter}
%\setmarginsrb{1in}{1in}{1in}{1in}{0pt}{0mm}{0pt}{0mm}

\begin{resume}
%
\section{\underline{EXPERIENCE}}
\vspace{1.0pc}
{\bf Software Engineer} - Google, Inc. Mountain View, CA
  \hfill \textbf{Sept. `14 - present}
  \begin{compactlist}
    \item Project Tango: Develop algorithms and applications for Visual-
      Inertial SLAM and sparse mapping.
  \end{compactlist}
  %
{\bf Senior Systems Engineer} - hiDOF, Inc., South San Francisco, CA
  \hfill \textbf{July `13 - Sept. `14}
  \begin{compactlist}
    \item Key software developer for major client. C++ system design, algorithm
      development, and build-system support.
    \item Projects in wheeled vehicle motion planning, visual inertial
      navigation, and visual monocular SLAM.
  \end{compactlist}
  %
{\bf Research Intern} - Willow Garage, Inc., Menlo Park, CA
  \hfill \textbf{Sept. `10 - June `13}
  \begin{compactlist}
    %\item Created systems, controllers, and user interfaces for teleoperated
    %  mobile manipulation.
    %\item Developed and maintained the first widely-used PR2 teleoperation
    %  application.
    %  , which saw extensive use by other researchers and formed a major
    %  component of the \textit{Robots for Humanity} project.
    \item Developed novel optimization-based controller and user interfaces for
      assisted collision-free teleoperation.
    \item Conducted user experiments and authored papers published in major
      robotics conferences.
    %\item Authored new ROS packages for 6-DOF/haptic user interaction and
      robot teleoperation.
  \end{compactlist}
  %
{\bf Graduate Researcher} - Salisbury Robotics Lab, Stanford, CA
  \hfill \textbf{Aug. `08 - June `13}
  \begin{compactlist}
    \item Developed new algorithms for haptic rendering and robot control
      (in collaboration with Willow Garage).
    \item Implemented miniature stereo camera sensor for a robot gripper
      (PCB design, mechanical hardware prototyping).
  \end{compactlist}
  %
\CVOnly{
{\bf Electrical Engineering Intern} - Qual-Tron, Inc., Tulsa, OK
  \hfill \textbf{March `06 - Feb. `07}
  \begin{compactlist}
    \item Designed and implemented test procedures for IR and magnetic sensor products.
    \item Led redesign of a magnetic sensor product to reduce cost and simplify assembly.
  \end{compactlist}
}
%
\hide{
  \\[0.4pc]{\bf Summer Intern} - Atmel Corporation
    \hfill \textbf{Summer `05}
  \\[0.4pc]{\bf Summer Research Intern} - NASA Glenn Research Center
    \hfill \textbf{Summer `04}%
}
%
{\bf \underline{Consulting:}}
  {\\[0.2pc]{\bf Motion Genesis, LLC}
    - Developed web-based visualization software for multi-body systems.
      \hfill \textbf{Spring `11 - Fall `13}}%
  {\\[0.0pc]{\bf Applied Materials, Inc.}
    - Subcontracting consultant for robot motion visualization.
      \hfill \textbf{Fall `12}}%
  \CVOnly{\\[0.0pc]{\bf Charm Labs}
    - Dynamics and control. Confidential.
      \hfill \textbf{Summer `12}}%
%
%
\section{\underline{SKILLS}}
\vspace{1.0pc}
%\textbf{Overview:} Strong expertise in robotics, dynamics, controls, and applied mathematics.
\textbf{Applied Math -} Expert in dynamics, kinematics, and 3D geometry as applied to robotics, simulation, and graphics.
\\[0.0pc]\textbf{Software Languages -} C++ (6 years) in large, complex projects featuring multi-threaded, event-driven, and multi-process designs, with a focus on quality and maintainability. Proficient in Python, Javascript, and MATLAB.
\\[0.0pc]\textbf{Software Tools -}
Expert knowledge of ROS. Experience with Eigen, OpenMP, MoveIt!, PCL, OpenCV, OpenGL, Qt.
Development in Ubuntu Linux (expert) and Windows (proficient)
using version control (git, svn) and issue tracking.
\\[0.0pc]\textbf{Electronics -} Circuit design/debugging, prototype PCB layout/fabrication, embedded systems.
\\[0.0pc]\textbf{Hardware -} General machine shop rapid-prototyping skills, and proficient in CAD tools (Solidworks).
%\\[0.0pc]\textbf{Other - } Speak/write English (native), Spanish (fluent), French (proficient). Private pilot, recording engineer, bassist.
\\[0.0pc]\textbf{Languages -} English (native), Spanish (fluent), French (proficient).
\\[0.0pc]\textbf{Other -} Private pilot, recording engineer, bassist.
%
%
\section{\underline{EDUCATION}}
\vspace{1.0pc}
\textbf{Ph.D.} Mechanical Engineering under Professor Ken Salisbury, Stanford University, 3.94 GPA
\hfill \textbf{June `13}%
{ {\small
%\\[0.0pc]{\bf Advisor:} Professor Kenneth Salisbury
\\[0.0pc]{\bf Thesis:} Robot Telemanipulation in Unstructured Environments: Sensors, Algorithms, Interfaces.
\\[-0.6pc]} }
%
\\[0.0pc]\textbf{M.S.} Mechanical Engineering, Stanford University, 3.97 GPA \hfill  \textbf{March `09}
\\[0.0pc]\textbf{B.S.} Engineering Physics, The University of Tulsa, 3.99 GPA \hfill \textbf{May `07}
%

%
%
\section{\underline{TEACHING}}
\vspace{1.0pc}
%{\bf Instructor: }
%\\[0.0pc]
{\bf Instructor:} ENGR 105 Controls, Stanford University, 72 students.
  \hfill \textbf{\CVOnly{Winter }2015}%
\\[0.0pc]
{\bf Instructor:} ENGR 14 Statics, Stanford University, 77 students.
  \hfill \textbf{\CVOnly{Spring }2014}%
\\[0.0pc]
{\bf Instructor:} ME 101 Dynamics, San Jose State University\ResumeOnly{, 50 students}\CVOnly{, 35 students}.
  \hfill \textbf{\ResumeOnly{2011, 2012,} \CVOnly{Fall }2013}%
%
\\[0.0pc]
\CVOnly{
\begin{tabularx}{\textwidth}{@{}l@{ }Xr@{}}
  {\bf Instructor:} & ME 101 Dynamics, San Jose State University\CVOnly{, 49 students}.
                    & \hfill \textbf{Fall 2012}%
	%\\[0.0pc]                  & Student-rated 4.8/5.0 for overall teaching quality. &
	\\[0.0pc]
  {\bf Instructor:} & ME 101 Dynamics, San Jose State University, 56 students.
                    & \hfill \textbf{Fall 2011}
	%\\[0.0pc]                  & Student-rated 4.6/5.0 for overall teaching quality. &
\end{tabularx}
}
%
{\bf Instructor:} Programming and Robotics, EPGY Summer Institutes at Stanford.
  \hfill \textbf{\CVOnly{Summer} 2010}%
%
\CVOnly{
  \\[0.4pc]
  \begin{tabularx}{\textwidth}{@{}l@{ }l@{ - }Xr@{}}
    Course Assistant: & ME 331b & Dynamics and Control with Paul Mitiguy.
    & \textbf{\CVOnly{Spring }2012}
  \\[0.0pc]
    Course Assistant: & CS 277 & Experimental Haptics with Ken Salisbury.
    & \textbf{\CVOnly{Winter }2011}
  \\[0.0pc]
    Course Assistant: & CS 223a & Robotics with Oussama Khatib.
    & \textbf{\CVOnly{Winter }2010}
  \\[0.0pc]
    Course Assistant: & ENGR 15 & Dynamics with Paul Mitiguy.
    & \textbf{\CVOnly{Fall }2009}
  \end{tabularx}
}
%
%
\ResumeOnly{\section{\underline{SELECTED PUBLICATIONS}}}
\CVOnly{\section{\underline{PUBLICATIONS}}}
\vspace{1.2pc}
%
%\CVOnly{\vspace{0.2pc}\textbf{Journal Articles}\\[0.4pc]}
%
%
\BothSmall{\boldName{A. Leeper}, K. Hsiao, M. Ciocarlie, I. Sucan, and K. Salisbury.
Methods for Collision-Free Arm Teleoperation in Clutter Using Constraints from 3D Sensor Data.
2013 International Conference on Humanoid Robots. October, 2013. Atlanta, Georgia. \\[0.4pc]}
%
\CVOnlySmall{\boldName{A. Leeper}, K. Hsiao, M. Ciocarlie, I. Sucan, K. Salisbury.
Assisted Arm Teleoperation in Clutter Using Constraints from 3D Sensor Data. In 2nd Workshop on Robots in Clutter:
Preparing robots for the real world (in conjunction with RSS). June 2013, Berlin, Germany. \\[0.4pc]}
%
\CVOnlySmall{Chen, Tiffany., Ciocarlie, Matei., Cousins, Steve., Grice, Phillip M.., Hawkins, Kelsey., Hsiao, Kaijen., Kemp, Charlie., King, ChihHung., Lazewatsky, Daniel., \boldName{Leeper, Adam Eric.}, Nguyen, Hai., Paepcke, Andreas., Pantofaru, Caroline., Smart, William., and
Takayama, Leila.
%Robots for Humanity: A Case Study in Assistive Mobile Manipulation.
Robots for humanity: using assistive robotics to empower people with disabilities.
IEEE Robotics and Automation Magazine
special issue on Assistive Robotics. Volume 20, Issue 1, 2013. \\[0.4pc]}
%
\CVOnlySmall{
  A. Pratkanis, \boldName{A. Leeper}, K. Salisbury.
  Replacing the Office Intern: An Autonomous Coffee Run with a Mobile Manipulator.
  ICRA, May 2013, Karlsruhe, Germany. \\[0.4pc]}
%
\CVOnlySmall{
  M. Ciocarlie, K. Hsiao, \boldName{A. Leeper}, D. Gossow.
  Mobile Manipulation Through an Assistive Home Robot.
  IROS, October 2012, Algarve, Portugal. \\[0.4pc]}
%
\BothSmall{
  \boldName{A. Leeper}, S. Chan, and K. Salisbury.
  Point Clouds Can Be Represented as Implicit Surfaces for Constraint-Based Haptic Rendering.
  ICRA, May 2012, St. Paul, MN. \\[0.4pc]}
%
\CVOnlySmall{
  \boldName{A. Leeper}, S. Chan, K. Hsiao, M. Ciocarlie, K. Salisbury.
  Constraint-based Haptic Rendering for Teleoperated Robot Grasping.
  IEEE Haptics Symposium, March 2012, Vancouver, Canada. \\[0.4pc]}
%
\BothSmall{
  \boldName{A. Leeper}, K. Hsiao, M. Ciocarlie, L. Takayama, D. Gossow.
  Strategies for Human-in-the-Loop Robotic Grasping.
  HRI, March 2012, Boston, MA. \\[0.4pc]}
%
\CVOnlySmall{
  R. Brewer, \boldName{A. Leeper}, K. Salisbury.
  A Friction Differential and Cable Transmission Design for a 3-DOF Haptic Device with Spherical Kinematics.
  IROS, Sept. 2011, San Francisco, CA. \\[0.4pc]}
%
\CVOnlySmall{
  D. Gossow, \boldName{A. Leeper}, D. Hershberger, M. Ciocarlie.
  Interactive Markers: 3-D User Interfaces for ROS Applications [ROS Topics].
  IEEE Robotics and Automation Magazine, December 2011. \\[0.4pc]}
%
\CVOnlySmall{
  \boldName{A. Leeper}, S. Chan, and K. Salisbury.
  Constraint-based 3-DOF Haptic Rendering of Arbitrary Point Cloud Data.
  RSS Workshop on RGB-D Cameras, June 2011, Los Angeles, CA. \\[0.4pc]}
%
\CVOnlySmall{
  \boldName{A. Leeper}, K. Hsiao, E. Chu, and K. Salisbury.
  Using Near-Field Stereo Vision for Robotic Grasping in Cluttered Environments.
  ISER, Dec. 2010, Delhi, India.  \CVOnlySmall{\\[0.4pc]} }
%
\CVOnlySmall{
  Caruso, John F; Hari, P; \boldName{Leeper, Adam E}; Coday, Michael A;
  Monda, Julie K; Ramey, Elizabeth S; Hastings, Lori P; Golden, Mallory R;
  Davison, Steve W.
  Impact of Acceleration on Blood Lactate Values Derived From High-Speed Resistance Exercise.
  Journal of Strength \& Conditioning Research. 23(7):2009-2014, October 2009.  \\[0.4pc]}
%
\CVOnlySmall{
  Caruso J.F., Hari P., Coday M.A., \boldName{Leeper A.}, Ramey E.S.,
  Monda J.K., Hastings L.P., and Davison S. (2008).
  Performance evaluation of a high-speed inertial exercise trainer.
  The Journal of Strength \& Conditioning Research. 22(6): 1760-1768. \\[0.4pc]}
%
\CVOnlySmall{
  \boldName{A. Leeper}, M. Coday, P. Hari, J. Caruso.
  Instrumentation of a High-Speed Inertial Exercise Device Using Load Cell Transducers.
  Proceedings of 53rd International Instrumentation Symposium, April 2007, Tulsa, OK.}
%
\CVOnly{
\section{\underline{PRESENTATIONS}}
\vspace{1.0pc}
\textbf{\underline{Invited Talks:}}\\[0.4pc]
``Telemanipulation using PCL."
PCL Tutorial at Robotics: Science and Systems 2011. Los Angeles, CA. July 2011. \\[0.4pc]
%
``Instrumentation of a High-Speed Inertial Exercise Device Using Load Cell Transducers."
ISA EXPO 2007. Houston, TX. October 2007. \\[0.4pc]
%
\textbf{\underline{Conference Presentations:}}\\[0.4pc]
%
``Assisted Arm Teleoperation in Clutter Using Constraints from 3D Sensor Data."
2nd Workshop on Robots in Clutter: Preparing Robots for the Real World (in conjunction with RSS).
Berlin, Germany. June 2013. \\[0.4pc]
%
``Point Clouds Can Be Represented as Implicit Surfaces for Constraint-Based Haptic Rendering."
International Conference on Robotics and Automation. St. Paul, MN. May 2012. \\[0.4pc]
%
``Constraint-based Haptic Rendering for Teleoperated Robot Grasping."
IEEE Haptics Symposium. Vancouver, Canada. March 2012. \\[0.4pc]
%
``Constraint-based 3-DOF Haptic Rendering of Arbitrary Point Cloud Data."
RGB-D: Advanced Reasoning with Depth Cameras (workshop in conjunction with RSS). Los Angeles, CA. June 2011. \\[0.4pc]
%
``Using Near-Field Stereo Vision for Robotic Grasping in Cluttered Environments."
International Sympoisum on Experimental Robotics. New Delhi, India. December 2010. \\[0.4pc]
%
``Instrumentation of a High-Speed Inertial Exercise Device Using Load Cell Transducers."
53rd International Instrumentation Symposium. Tulsa, OK. April 2007.
%
}
%
%
\CVOnly{
\section{\underline{OPEN SOURCE SOFTWARE}~(github.com/aleeper)}
\vspace{1.2pc}
\begin{tabularx}{\textwidth}{@{}l@{ - }Xr@{}}
MGView & Javascript web app for visualizing rigid body simulations. Author and maintainer.%
\\[0.0pc]
ROS & Contributor and maintainer of packages in the visualization and device driver stacks.%
\\[0.0pc]
MoveIt! & Contributor to the user interaction and visualization tools within MoveIt! %
\\[0.0pc]
three.js & Contributed STL parser module to enable importing of CAD parts (e.g. from SolidWorks).%
\end{tabularx}
}
%
%
%
\CVOnly{
\section{\underline{AWARDS}}
\vspace{1.0pc}
          2007-2012 National Science Foundation Graduate Research Fellowship
\CVOnly{\\[0.0pc]2007 Stanford Graduate Fellowship }
\CVOnly{\\[0.0pc]2007 John McCamey Award presented by ISA}
\ResumeOnly{\\[0.0pc] Member of Tau Beta Pi and Phi Kappa Phi Honor Societies}
\CVOnly{
\\[0.0pc] Member, Tau Beta Pi Engineering Honor Society
\\[0.0pc] Member, Sigma Pi Sigma Physics Honor Society
\\[0.0pc] Member, Phi Kappa Phi Honor Society
\\[0.0pc] Member, Mortar Board National College Senior Honor Society
}
}
%
\CVOnly{
%\clearpage
\section{\underline{REFERENCES}}
\vspace{1.0pc}
Dr. Kenneth Salisbury, Professor (Research) of Computer Science, 650.465.5700, jks@robotics.stanford.edu
\\[0.0pc]Dr. Paul Mitiguy, Motion Genesis LLC and Stanford University Lecturer, 650.346.9595,  mitiguy@stanford.edu
\\[0.0pc]Dr. Kaijen Hsiao, Bosch Research and Technology Center, 617.304.1759, kaijenhsiao@gmail.com
%\\[0.0pc]Oussama Khatib, Prof. Computer Science, 650.723.9753, khatib@cs.stanford.edu
}
%
\end{resume}
\end{document}







