% LaTeX resume using res.cls
\documentclass[line,margin]{res}
\usepackage{ifthen}
\usepackage{tabularx,booktabs}
%\usepackage{helvetica} % uses helvetica postscript font (download helvetica.sty)
%\usepackage{newcent}   % uses new century schoolbook postscript font
\setlength{\textheight}{9.25in}
\setlength{\voffset}{-0.125in}

\newsectionwidth{20pt}

\newcommand{\ifthen}[2]{\ifthenelse{#1}{#2}{}}

%%% We can put stuff inside a \CVOnly{} tag to make it only show up for a CV :)
\providecommand{\ResumeCV}{0}% fallback definition % 0=Resume, 1=CV
%\def\ResumeCV{0}  % 0=Resume, 1=CV
\ifthen { \ResumeCV = 0 }
        { \newcommand{\ResumeOnly}[1]{#1}
          \newcommand{\CVOnly}[1]{}
          \newcommand{\CVOnlySmall}[1]{{\small \CVOnly{#1}}}
          \newcommand{\ResumeOnlySmall}[1]{{\small \CVOnly{#1}}}
          \newcommand{\BothSmall}[1]{{\small #1}}   }
\ifthen { \ResumeCV = 1 }
        { \newcommand{\ResumeOnly}[1]{}
          \newcommand{\CVOnly}[1]{#1}
          \newcommand{\CVOnlySmall}[1]{{\CVOnly{#1}}}
          \newcommand{\ResumeOnlySmall}[1]{{\CVOnly{#1}}}
          \newcommand{\BothSmall}[1]{{#1}}    }

\newcommand{\hide}[1]{}



\newcommand{\boldName}[1]{\textbf{#1}}

%\setpapersize{USletter}
%\setmarginsrb{1in}{1in}{1in}{1in}{0pt}{0mm}{0pt}{0mm}

\begin{document}

\name{Adam E. Leeper}
% \address used twice to have two lines of address
\address{469 Homer Ave, Palo Alto, CA 94301}
\address{650.762.6844}
\address{adamleeper@gmail.com}
\address{www.adamleeper.com}
%\address{www.stanford.edu/$\sim$aleeper }


\begin{resume}

\section{\underline{EDUCATION}}
\vspace{1.0pc}
Ph.D. Mechanical Engineering, Stanford University\ResumeOnly{, Advisor Ken Salisbury}
\hfill 2013%
\CVOnly{ {\small
\\[0.0pc]{\bf Advisor:} Professor Kenneth Salisbury
\\[0.0pc]{\bf Thesis:} Robot Telemanipulation in Unstructured Environments: Sensors, Algorithms, Interfaces.
\\[-0.6pc]} }
%
\\[0.0pc]M.S. Mechanical Engineering, Stanford University, 3.97 GPA \hfill  2009
%\\[0.0pc]{\bf University of Tulsa: }
\\[0.0pc] B.S. Engineering Physics, The University of Tulsa, 3.99 GPA \hfill 2007
%
%
\section{\underline{EXPERIENCE}}
\vspace{1.0pc}
{\bf Senior Systems Engineer} - hiDOF, Inc., South San Francisco, CA  \hfill 2013 - 2014
\\[0.0pc]Software engineering  and technology transfer for robotics applications.
%
\\[0.4pc]{\bf Research Intern} - Willow Garage, Inc., Menlo Park, CA \hfill 2010 - 2013
\\[0.0pc]Created systems, controllers, and user interfaces for teleoperated mobile manipulation.
%\CVOnly{\\[0.0pc]Created systems and user interfaces for teleoperated mobile manipulation.}
%
\\[0.4pc]{\bf Research Assistant} - Salisbury Robotics Lab, Stanford, CA \hfill 2008 - 2013
\\[0.0pc]Conducted research in algorithms for haptic rendering and robot control.
\CVOnly{\\[0.0pc]Implemented miniature stereo camera sensor hardware for a robot gripper.}
%
\CVOnly{
\\[0.4pc]{\bf Electrical Engineering Intern} - Qual-Tron, Inc., Tulsa, OK \hfill 2006 - 2007
\\[0.0pc] Designed and implemented test procedures for IR and magnetic sensor products.
\\[0.0pc]Led redesign of a magnetic sensor product to reduce cost and simplify assembly.}
%
\hide{
\\[0.4pc]{\bf Summer Intern} - Atmel Corporation \hfill 2005
\\[0.4pc]{\bf Summer Research Intern} - NASA Glenn Research Center \hfill 2004%
}
%
\\[0.4pc]{\bf \underline{Consulting:}}
\\[0.2pc]{\bf Motion Genesis, LLC} - Developed visualization tools for multi-body systems. \hfill 2011-2013
\\[0.0pc]{\bf Applied Materials, Inc.} - Subcontracting consultant for robot motion visualization. \hfill 2012
\\[0.0pc]{\bf Charm Labs} - Dynamics and control. Confidential. \hfill 2012
%

%
%
\section{\underline{TEACHING}}
\vspace{1.0pc}
%{\bf Instructor: }
%\\[0.0pc]
{\bf Instructor:} ENGR 14 Statics, Stanford University\CVOnly{, 77 students}. \hfill \CVOnly{Spring }2014%
\\[0.0pc]
{\bf Instructor:} ME 101 Dynamics, San Jose State University\CVOnly{, 35 students}. \hfill \ResumeOnly{2011, 2012,} \CVOnly{Fall }2013%
%
\CVOnly{
\\[0.0pc]
\begin{tabularx}{\textwidth}{@{}l@{ }Xr@{}}
	         {\bf Instructor:} & ME 101 Dynamics, San Jose State University\CVOnly{, 49 students}. & \hfill Fall 2012%
	\\[0.0pc]                  & Student-rated 4.8/5.0 for overall teaching quality. &
	\\[0.0pc]{\bf Instructor:} & ME 101 Dynamics, San Jose State University, 56 students. & \hfill Fall 2011
	\\[0.0pc]                  & Student-rated 4.6/5.0 for overall teaching quality. &
\end{tabularx}
}
%
\\[0.0pc]
{\bf Instructor:} Programming and Robotics, EPGY Summer Institutes at Stanford. \hfill \CVOnly{Summer}
2010
%
\\[0.4pc]
\begin{tabularx}{\textwidth}{@{}l@{ }l@{ - }Xr@{}}
Course Assistant: & ME 331b & Dynamics and Control with Paul Mitiguy. 
& \CVOnly{Spring }2012
\\[0.0pc]
Course Assistant: & CS 277 & Experimental Haptics with Ken Salisbury. 
& \CVOnly{Winter }2011
\\[0.0pc]
Course Assistant: & CS 223a & Robotics with Oussama Khatib. 
& \CVOnly{Winter }2010
\\[0.0pc]
Course Assistant: & ENGR 15 & Dynamics with Paul Mitiguy. 
& \CVOnly{Fall }2009
\end{tabularx}
%
%
\ResumeOnly{\section{\underline{SELECTED PUBLICATIONS}}}
\CVOnly{\section{\underline{PUBLICATIONS}}}
\vspace{1.2pc}
%
%\CVOnly{\vspace{0.2pc}\textbf{Journal Articles}\\[0.4pc]}
%
%
\BothSmall{\boldName{A. Leeper}, K. Hsiao, M. Ciocarlie, I. Sucan, and K. Salisbury.
Methods for Collision-Free Arm Teleoperation in Clutter Using Constraints from 3D Sensor Data.
2013 IEEE-RAS International Conference on Humanoid Robots. October, 2013. Atlanta, Georgia. \\[0.4pc]}
%
\CVOnlySmall{\boldName{A. Leeper}, K. Hsiao, M. Ciocarlie, I. Sucan, K. Salisbury.
Assisted Arm Teleoperation in Clutter Using Constraints from 3D Sensor Data. In 2nd Workshop on Robots in Clutter:
Preparing robots for the real world (in conjunction with RSS). June 2013, Berlin, Germany. \\[0.4pc]}
%
\CVOnlySmall{Chen, Tiffany., Ciocarlie, Matei., Cousins, Steve., Grice, Phillip M.., Hawkins, Kelsey., Hsiao, Kaijen., Kemp, Charlie., King, ChihHung., Lazewatsky, Daniel., \boldName{Leeper, Adam Eric.}, Nguyen, Hai., Paepcke, Andreas., Pantofaru, Caroline., Smart, William., and
Takayama, Leila.
%Robots for Humanity: A Case Study in Assistive Mobile Manipulation.
Robots for humanity: using assistive robotics to empower people with disabilities.
IEEE Robotics and Automation Magazine
special issue on Assistive Robotics. Volume 20, Issue 1, 2013. \\[0.4pc]}
%
\CVOnlySmall{A. Pratkanis, \boldName{A. Leeper}, K. Salisbury. Replacing the Office Intern: An Autonomous Coffee Run with a Mobile Manipulator. ICRA, May 2013, Karlsruhe, Germany. \\[0.4pc]}
%
\CVOnlySmall{M. Ciocarlie, K. Hsiao, \boldName{A. Leeper}, D. Gossow. Mobile Manipulation Through an Assistive Home Robot. IROS, October 2012, Algarve, Portugal. \\[0.4pc]}
%
\BothSmall{\boldName{A. Leeper}, S. Chan, and K. Salisbury. Point Clouds Can Be Represented as Implicit Surfaces for Constraint-Based Haptic Rendering. ICRA, May 2012, St. Paul, MN. \\[0.4pc]}
%
\CVOnlySmall{\boldName{A. Leeper}, S. Chan, K. Hsiao, M. Ciocarlie, K. Salisbury. Constraint-based Haptic Rendering for Teleoperated Robot Grasping. IEEE Haptics Symposium, March 2012, Vancouver, Canada. \\[0.4pc]}
%
\BothSmall{\boldName{A. Leeper}, K. Hsiao, M. Ciocarlie, L. Takayama, D. Gossow. Strategies for Human-in-the-Loop Robotic Grasping. HRI, March 2012, Boston, MA. \\[0.4pc]}
%
\CVOnlySmall{R. Brewer, \boldName{A. Leeper}, K. Salisbury. A Friction Differential and Cable Transmission Design for a 3-DOF Haptic Device with Spherical Kinematics. IROS, Sept. 2011, San Francisco, CA. \\[0.4pc]}
%
\CVOnlySmall{D. Gossow, \boldName{A. Leeper}, D. Hershberger, M. Ciocarlie. Interactive Markers: 3-D User Interfaces for ROS Applications [ROS Topics]. IEEE Robotics and Automation Magazine, December 2011. \\[0.4pc]}
%
\CVOnlySmall{\boldName{A. Leeper}, S. Chan, and K. Salisbury. Constraint-based 3-DOF Haptic Rendering of Arbitrary Point Cloud Data. RSS Workshop on RGB-D Cameras, June 2011, Los Angeles, CA. \\[0.4pc]}
%
\BothSmall{\boldName{A. Leeper}, K. Hsiao, E. Chu, and K. Salisbury. Using Near-Field Stereo Vision for Robotic Grasping in Cluttered Environments. ISER, Dec. 2010, Delhi, India.  \CVOnlySmall{\\[0.4pc]} }
%
\CVOnlySmall{Caruso, John F; Hari, P; \boldName{Leeper, Adam E}; Coday, Michael A; Monda, Julie K; Ramey, Elizabeth S; Hastings, Lori P; Golden, Mallory R; Davison, Steve W. Impact of Acceleration on Blood Lactate Values Derived From High-Speed Resistance Exercise. Journal of Strength \& Conditioning Research. 23(7):2009-2014, October 2009.  \\[0.4pc]}
%
\CVOnlySmall{Caruso J.F., Hari P., Coday M.A., \boldName{Leeper A.}, Ramey E.S., Monda J.K., Hastings L.P., and Davison S. (2008). Performance evaluation of a high-speed inertial exercise trainer. The Journal of Strength \& Conditioning Research. 22(6): 1760-1768. \\[0.4pc]}
%
\CVOnlySmall{\boldName{A. Leeper}, M. Coday, P. Hari, J. Caruso.
Instrumentation of a High-Speed Inertial Exercise Device Using Load Cell Transducers.
Proceedings of 53rd International Instrumentation Symposium, April 2007, Tulsa, OK.}
%
\CVOnly{
\section{\underline{PRESENTATIONS}}
\vspace{1.0pc}
\textbf{\underline{Invited Talks:}}\\[0.4pc]
``Telemanipulation using PCL."
PCL Tutorial/Workshop at Robotics: Science and Systems 2011. Los Angeles, CA. July 2011. \\[0.4pc]
%
``Instrumentation of a High-Speed Inertial Exercise Device Using Load Cell Transducers."
ISA EXPO 2007. Houston, TX. October 2007. \\[0.4pc]
%
\textbf{\underline{Conference Presentations:}}\\[0.4pc]
%
``Assisted Arm Teleoperation in Clutter Using Constraints from 3D Sensor Data."
2nd Workshop on Robots in Clutter: Preparing Robots for the Real World (in conjunction with RSS).
Berlin, Germany. June 2013. \\[0.4pc]
%
``Point Clouds Can Be Represented as Implicit Surfaces for Constraint-Based Haptic Rendering."
International Conference on Robotics and Automation. St. Paul, MN. May 2012. \\[0.4pc]
%
``Constraint-based Haptic Rendering for Teleoperated Robot Grasping."
IEEE Haptics Symposium. Vancouver, Canada. March 2012. \\[0.4pc]
%
``Constraint-based 3-DOF Haptic Rendering of Arbitrary Point Cloud Data."
RGB-D: Advanced Reasoning with Depth Cameras (workshop in conjunction with RSS). Los Angeles, CA. June 2011. \\[0.4pc]
%
``Using Near-Field Stereo Vision for Robotic Grasping in Cluttered Environments."
International Sympoisum on Experimental Robotics. New Delhi, India. December 2010. \\[0.4pc]
%
``Instrumentation of a High-Speed Inertial Exercise Device Using Load Cell Transducers."
53rd International Instrumentation Symposium. Tulsa, OK. April 2007.
%
}
%
\section{\underline{SKILLS}}
\vspace{1.0pc}Strong expertise in robotics, dynamics, controls, and applied mathematics.
%{\sl Computers: } Extensive experience operating in Linux and Windows.
\\[0.25pc] {\sl Computation: }
Comfortable in Linux and Windows environments.
Software engineering (C++, Python) for robotics and simulation, with extensive use of version control and issue tracking.
Proficiency in MATLAB for computation and data analysis.
Experience with ROS, Qt, PCL, OpenGL, OpenCV.
\\[0.25pc]{\sl Electronics: } Circuit design/debugging, prototype PCB layout/fabrication, embedded systems.
\\[0.25pc]{\sl Hardware: } General machine shop rapid-prototyping skills, and proficient in CAD tools (Solidworks).
\\[0.25pc]{\sl Languages: } English (native), Spanish (fluent), French (proficient reading and writing).
\\[0.25 		pc]{\sl Other: } Private pilot, recording engineer, bassist.
%
%
\CVOnly{
\section{\underline{AWARDS}}
\vspace{1.0pc}
          2007-2012 National Science Foundation Graduate Research Fellowship
\CVOnly{\\[0.0pc]2007 Stanford Graduate Fellowship }
\CVOnly{\\[0.0pc]2007 John McCamey Award presented by ISA}
\ResumeOnly{\\[0.0pc] Member of Tau Beta Pi and Phi Kappa Phi Honor Societies}
\CVOnly{
\\[0.0pc] Member, Tau Beta Pi Engineering Honor Society
\\[0.0pc] Member, Sigma Pi Sigma Physics Honor Society
\\[0.0pc] Member, Phi Kappa Phi Honor Society
\\[0.0pc] Member, Mortar Board National College Senior Honor Society
}
}
%
\CVOnly{
%\clearpage
\section{\underline{REFERENCES}}
\vspace{1.0pc}
Dr. Kenneth Salisbury, Prof. Computer Science, 650.465.5700, jks@robotics.stanford.edu
\\[0.0pc]Dr. Paul Mitiguy, Prof. Mechanical Engineering, 650.346.9595,  mitiguy@stanford.edu
\\[0.0pc]Dr. Kaijen Hsiao, Bosch Research and Technology Center, 617.304.1759, kaijenhsiao@gmail.com
%\\[0.0pc]Oussama Khatib, Prof. Computer Science, 650.723.9753, khatib@cs.stanford.edu
}
%
\end{resume}
\end{document}







